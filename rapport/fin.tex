\section{Mots de fin}
Pour finir ce rapport, nous aimerions parler de ce que nous avons appris, ce qui nous a plu dans ce projet, et ce qui nous a déplu.
\subsection{Ce que nous avons appris}
Nous n'avons pas tant appris sur le plan technique comparé à nos camarades,
cependant, cela peut s'expliquer par notre parcours. En effet, l'un a fait une licence informatique, tandis que l'autre a fait un DUT Informatique.
Nous avions donc déjà un fort bagage technique. Néanmoins, ce projet nous a tout de même permis de consolider nos connaissances,
et même d'apprendre de toutes nouvelles choses. Par exemple, nous avons appris que pour déclarer une fonction sans paramètres,
il fallait la déclarer comme ceci : \verb|type nom(void);|. En effet, si
nous ne mettons pas le \verb|void|, cela n'est plus un prototype de fonction.
Cela peut donc causer plusieurs soucis au niveau de la compilation (incohérences \verb|.h| et \verb|.c|).
Pour détecter ce type d'erreurs, nous avons donc appris qu'il fallait utiliser le tag gcc \verb|-Wstrict-prototypes|.
De plus, nous nous sommes efforcés d'utiliser des principes de programmation en C plutôt obscurs,
tels que ne pas utiliser d'assert dans le code en production, car avec un tag gcc, on peut désactiver les assert.
Nous avons donc défini une macro qui permet de vérifier si un malloc a bien alloué de la mémoire, sinon, le programme plante.

\subsection{Les + du projet}
On était relativement libre sur le projet. On pouvait implémenter une solution algorithmique différente de celle proposée,
et nous avions un minimum à implémenter, mais pas un maximum. De plus, la seule structure de code vraiment imposée est
celle de départ (même si cela nous a bien embêté). Aussi, nous étions bien encadrés,
dès que nous avions une question, l'enseignant nous répondait de façon approfondie et compréhensible.
De plus, les solutions algorithmiques proposées étaient intéressantes, voire intriguantes.
Enfin, le sujet était intéressant dans son ensemble.

\subsection{Les - du projet}
Les fichiers \verb|.c| et \verb|.h| imposés étaient plutôt embêtants.
Nous en avons déjà parlé plus haut \textsuperscript{\ref{ssec:module-util}}, mais cela nous a forcé à faire un fichier 
\verb|util.h| alors que nous aurions pu mettre certains types dans 
\verb|neighbors.h| ou \verb|geometry.h|. Ensuite, le fait qu'il n'y ait pas 
d'\emph{Issue Board} sur la forge nous a forcé à créer un trello, ce n'est
pas si grave, mais tout de même embêtant.