\section{Choix techniques}
Durant ce projet, nous avons copieusement abusé des mallocs. Cela pose
certains soucis de gestion de la mémoire (principalement des \emph{memory leaks})
détectables avec \verb|valgrind|, mais nous avantage grandement.
En effet, cela nous permet de faire des
objets "persistents". Ils restent en mémoire jusqu'à la fin du programme
ou libération de la mémoire. Cela nous donne l'option d'utiliser des
structures de données génériques. Une structure de données

\subsection{Modularité}
Nous avons séparé notre projet en plusieurs modules, cela nous permet d'avoir une meilleure organisation au niveau des fichiers
ainsi que dans la logique des composants. Nos structures de données ne sont utilisable qu'a l'aide d'interfaces ce qui permet de changer
l'implémention sans perturber les autres modules. 


