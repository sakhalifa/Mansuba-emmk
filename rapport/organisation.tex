\section{Choix techniques}
Durant ce projet, nous avons copieusement abusé des mallocs. Cela pose
certains soucis de gestion de la mémoire (principalement des \emph{memory leaks})
détectables avec \verb|valgrind|, mais cela nous avantage grandement.
En effet, cela nous permet de faire des
objets "persistents". Ils restent en mémoire jusqu'à la fin du programme
ou libération de la mémoire. Cela nous donne l'option d'utiliser des
structures de données génériques. Une structure de données

\subsection{Modularité}
Nous avons séparé notre projet en plusieurs modules, ce qui nous permet d'avoir une meilleure organisation au niveau des fichiers
ainsi que dans la logique des composants. Nos structures de données ne sont utilisables qu'à l'aide d'interfaces ce qui permet de changer
l'implémention sans perturber les autres modules. 


\subsection{Structure du jeu}

\begin{minted}{c}
typedef struct {
    uint turn;
    uint max_turns;
    player_t *current_player;
    enum victory_type victory_type;
    struct world_t *world;
    array_list_t *captured_pieces_list;
    array_list_t *starting_position;
} game_t;
\end{minted}


Notre structure game\_t comprend tous les éléments nécessaires pour le déroulement
d'une partie du jeu.
Avec le recul et l'avancement dans le projet une amélioration que nous voulions mettre en place mais n'avons
pas eu le temps était de remplacer les champs captured\_pieces\_list et starting\_pos par des tableaux d'array\_list.
Les tableaux seraient de taille nombre\_de\_couleurs afin d'avoir une array\_list pour chaque couleur ce qui permet
une séparation entre les pièces capturées selon leurs couleurs. Le bénéfice d'un tel changement est 
d'éviter les parcours inutiles, par exemple, lors de la libération de pièces, nous devons dans un premier temps
parcourir la liste des pièces capturées afin de récupérer seulement celles de la couleur du joueur alors
qu'avec la nouvelle implémentation cette étape n'est pas nécessaire donc il y aurait un gain de temps et
une simplification du code.  


\subsection{Implémentation des structures de données}
Premièrement toutes nos structures de données (sauf l'arbre par manque de temps)
n'exportent pas l'implémentation de la structure sous-jacente. Dans le \verb|.h|, il n'y a que la déclaration
du nom de la structure, pas son implémentation. Cela nous permet de réaliser un couplage faible
entre les différents fichiers utilisant ces structures de données.
\subsubsection{Tableau dynamique}
Pour le tableau dynamique, nous avons besoin des opérations standards 
sur une liste, à savoir l'ajout, la suppression, l'insertion, l'affectation, et la récupération.
Vu que c'est un tableau sous-jacent, l'insertion et la suppression seront en complexité en temps O(n).
Il faut déplacer tous les éléments dans le tableau.
Ensuite, la récupération et affectation seront en complexité en temps O(1).
Enfin, l'ajout d'un élément à la fin du tableau sera en temps moyen O(log(n)).
En effet, lorsqu'on dépasse la taille du tableau, on doit allouer un nouveau tableau
pouvant contenir tous les éléments précédents et l'élément à ajouter.
L'approche naïve serait d'allouer un nouveau tableau de taille n+1. Le problème,
c'est que si on veut rajouter pas 1, mais 2 élements, on devra allouer 2 nouveaux tableaux, et donc recopier la liste complète 2 fois.
L'approche un peu plus subtile (celle qu'on a choisie), c'est de doubler la taille du tableau.
Cela évitera de devoir copier tous les éléments de la liste trop souvent. Au désavantage de gâcher un peu plus de mémoire.
\subsubsection{Arbre}
Pour notre arbre, nous avons choisi de l'implémenter avec une structure récursive.
Un noeud possède une valeur, une liste d'enfants et un parent.
Nous avons opté pour un double chaînage car nous en avions besoin pour remonter l'arbre
lors du calcul des mouvements possibles\textsuperscript{\ref{fig:example-moves}}.
Pour la liste des enfants, nous avons utilisé le tableau dynamique défini précédemment.
\subsubsection{Liste chaînée}
Nous avions aussi besoin d'une liste chaînée.
En effet, contrairement au tableau dynamique, l'affectation et la récupération sont en complexité O(n).
Mais, les opérations d'insertion et de suppression sont en O(1). En gardant
des pointeurs vers le dernier et premier élément de la liste, cela nous permet d'avoir un type
qui peut fonctionner comme une pile et une file en temps constant. Nous avions
particulèrement besoin d'une file pour le parcours en largeur\textsuperscript{\ref{alg:BFS}}.
\subsection{Structure des fichiers}
