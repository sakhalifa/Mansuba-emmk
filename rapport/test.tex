\section{Tests}

Tous les tests du projet sont rassemblés dans le dossier tst/ . 

\subsection{Structures de données}
Pour chacune des structures de données que nous avons utilisées (arbres, array\_list, liste chaînée),
il existe un fichier de test correspondant. Cela nous permet de nous assurer que ces structures, qui
forment la base de notre projet, fonctionnent sans problèmes. Dans ces tests, nous nous assurons que les fonctions
donnent le résultat attendu dans les cas classiques ainsi que dans des cas limites tel que la suppression 
dans une liste vide. 

Ces tests sont également examiné par \emph{valgrind} afin d'être sûr qu'il n'y ait pas 
de fuite de mémoire.

\subsection{Modules principaux}
Pour pouvoir tester nos modules principaux, on met en place une situation spécifique puis nous appelons une des 
fonctions du module et on compare cette nouvelle situation après l'éxécution à celle qui est attendu.

Certains modules n'ont cependant pas de test car difficilement réalisable. C'est le cas, par exemple, de notre
module player\_handler, qui s'occupe de gérer l'interaction d'un joueur à travers le terminal, vu que simuler
les entrées n'est pas trivial et surtout que c'est un module de haut niveau donc presque aucun module ne dépend de lui.

